\chapter{总结与展望}
\label{sec:summary}

文\cite{du2018hermite}中作者针对双曲守恒律,
基于五阶的HWENO和WENO重构以及广义黎曼问题解法器,
设计了一种两步四阶数值格式。
这种数值格式有一维和二维两个版本。
其中,
一维两步四阶数值格式采用五阶的HWENO重构函数值,
同时采用五阶的WENO重构导数值;二维两步四阶数值格式采用逐维的HWENO5重构函数值,
然后在两个空间方向上分别使用WENO5和HWENO5重构来获得导数值。
为了进一步提高数值格式的紧致性,
本文针对双曲守恒律,
开展了基于紧致埃尔米特重构的显式两步四阶有限体积格式的研究。
主要工作分为以下三个部分:
\begin{enumerate}
      \item 我们在\cref{sec:framework}中提出了改进的两步四阶时间推进框架。
            首先,
            我们在研究中发现原始的两步四阶时间推进框架的紧致性和高精度不可兼得。
            如果导数重构采用更加紧致的埃尔米特重构,
            那么所得格式在光滑区域只有三阶精度。
            随后,
            我们分析了其原因,
            并在此基础上提出了改进的两步四阶时间推进框架。
            在改进的框架中导数重构采用更加紧致的埃尔米特重构时,
            获得的数值格式可以在光滑区域保持四阶精度。
      \item 我们在\cref{sec:1D-method}中设计了一维基于紧致埃尔米特重构的基本无振荡的两步四阶数值格式。
            首先,
            我们先构造了不同精度的线性埃尔米特重构,
            并设计了相应的线性格式。
            然后我们做了线性稳定性分析,
            在空间四阶、六阶和八阶精度下,
            分别得到了一个线性稳定的两步四阶数值格式。
            接着,
            我们构造了两种四阶精度基本无振荡的非线性重构,
            即加权型和杂交选择型的紧致埃尔米特重构,
            并设计了相应的时空四阶精度基本无振荡的两步四阶格式。
            两种格式均被推广到了空间八阶精度。
            过后,
            我们讨论了数值格式的紧致性,
            得出了结论:与文\cite{du2018hermite}中的一维两步四阶格式相比,
            我们的一维四阶精度两步四阶格式更加紧致。
            最后我们在\cref{sec:1D-examples}中给了许多算例,
            验证了我们的一维数值格式具有高精度、稳定、紧致、高效以及基本无振荡的优良特性。
      \item 我们在\cref{sec:2D-method}中设计了二维基于紧致埃尔米特重构的基本无振荡的两步四阶数值格式。
            首先,
            我们经过对模板的精心挑选,
            成功得到真正的二维线性紧致埃尔米特重构,
            并设计了相应的线性格式。
            这些二维重构在一维问题中可以退化到我们先前构造的一维重构,
            从而这些二维重构可以取得类似一维重构的良好性能。
            然后,
            基于二阶和四阶的线性重构,
            我们构造了两种四阶精度基本无振荡的非线性重构,
            即加权型和杂交选择型的紧致埃尔米特重构,
            并设计了相应的时空四阶精度基本无振荡的两步四阶格式。
            接下来,
            我们将两种格式推广到了空间八阶精度,
            并且由于这些数值格式是真正二维的,
            有推广到非结构化网格的潜力。
            接着,
            我们讨论了数值格式的紧致性,
            得出了结论:与文\cite{du2018hermite}中的二维两步四阶格式相比,
            我们的二维四阶精度两步四阶格式更加紧致。
            最后我们在\cref{sec:2D-examples}中给了许多算例,
            验证了我们的二维数值格式具有高精度、稳定、紧致、高效以及基本无振荡的优良特性。
\end{enumerate}

在本文研究的方向内,
我们还有很多在未来的工作中值得继续努力的地方:
\begin{enumerate}
      \item 本文完成了欧拉结构网格的数值格式的设计,
            我们希望能够将之推广到非结构网格,
            并希望能进一步推广到拉格朗日方法和任意拉格朗日-欧拉方法。
      \item 本文完成了一维和二维正交网格的数值格式的设计,
            我们希望能够将之推广到三维正交网格情形,
            以及二维柱(球)坐标系、三维球坐标系等带源项的流体力学欧拉方程组情形。
      \item 除此之外,
            我们还希望能够进一步设计具有更多物理性质的数值格式,
            例如:保正性和熵不等式等。
\end{enumerate}
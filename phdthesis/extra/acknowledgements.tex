\chapter{致~~~~谢}

在论文即将完成之际,
回想起这五年的博士研究生期间,
发生了诸多幸运与不幸。
想起二十一年的学生时代将随着博士毕业而彻底结束,
感慨万千,
却欲说还休。

这五年间不幸有三,
其一是最爱我的爷爷去世了,
2019年9月初。
除了在高中的三年需要住校以及在外地读大学的四年,
爷爷几乎始终陪在我身边,
他总是能容忍我的任性、总是能容忍我的顽皮,
不停地给我讲他的军旅往事、给我讲他的工厂轶闻、给我讲数学物理、给我讲天文地理、给我讲政治历史、给我讲人生的道理,
即使上了一天班、即使刚和奶奶吵架、即使身患重病。
然而,
他永远的离开了。
我仍记得,
和我讲过的最后一句话:“去北京读书吧,
安慰爷爷一句:‘死亡没什么大不了的,
爷爷,
你勇敢一点。
’”现在好想和爷爷讲一句:“我博士要毕业了,
你看我厉害不,
……,
我再也不是小孩子了,
之前我太任性了,
我太顽皮了,
……,
我好想让你看我成家立业的样子。
”特在此思念您。

其二是我敬爱的李老师去世了,
2023年5月中旬。
最初得知消息是在微信群中,
师兄说李老师走了,
我起初以为是从首师大离开了,
然后觉得师兄在开玩笑,
始终不敢相信李老师竟然英年早逝;直到遗体告别仪式,
我才肯接受这个事实。
李老师在学术上学识渊博、严谨治学、言传身教,
曾为了帮助我理解掌握广义黎曼问题解法器,
让我连续数周向他报告同样的文章,
直到确认我真的理解清楚了为止。
在撰写文章时,
不厌其烦的阅读不才写的草稿,
给出中肯的意见,
帮助修改润色文章的措辞,
力求严谨、细致。
在生活上,
也经常关心晚辈的生活,
教会晚辈人生的道理。
哎,
天妒英才,
您无私的教诲我竟无以为报。
特在此怀念您。

其三是三年的新冠疫情,
是为此奉献的广大人民。
2020年年初,
本以为是学生时代习以为常的寒假,
结果9月初才得以返校。
面对未知的病毒爆发,
除了尽力帮助自己和身边家人老师朋友、用科学指导行动尽可能保证自己和大家的安全,
我无能为力。
多亏广大群众的努力,
尤其是医疗工作者以及社区工作人员等人全力保障医疗和生存物资的供应。
在此感谢这些伟大的人们。

正是不幸遇的多了,
这些年间的幸运才变得弥足珍贵。
首先要感谢成老师的知遇之恩。
最初,
在2018年春夏之交,
在南航听了成老师对九所的宣讲,
我便结下了与成老师的缘分,
也结下了与九所的五年之缘。
于同年夏天,
南京大学主办的一个学术会议上,
有幸和成老师交流了研究生期间的规划。
在李老师换单位之后,
又是成老师接收了我。
我很清楚接替其他老师培养一个进行了一半科研训练的学生很难,
要花费更多的精力和耐心。
正是我很清楚这一点,
我对成老师的感激无以言表。
成老师深耕计算数学领域多年,
有着丰富的前沿视野和多年的科研经验,
十分感谢成老师帮助我在领域内开拓视野。
在学术文章和毕业论文撰写过程中,
提出了许多建设性意见,
投入了超多的心血和精力,
帮助我不断完善论文,
才使得不才的论文写成如今的模样。
在生活上,
成老师关照有加,
经常投喂水果、零食,
会关心我生活上的琐事,
关心我的状态,
关心我找工作等等。
总之,
十分感谢成老师在各个方面的帮助!我将要竭尽全力报答您的恩情。

第二个幸运是舒老师愿意指导不才的工作,
在数值实验的设计和文章撰写上提供了不可估量的帮助,
对此我十分感谢!早在大四的时候,
就听闻舒老师的大名,
在入门CFD时学习的WENO格式便是舒老师1997年的一份科技报告。
何尝敢想能和舒老师合作发表论文。

第三个幸运是临近毕业,
第两篇文章顺利接收了。
我要感谢两篇文章的编辑和给予我反馈意见的同行专家!他们的宝贵意见和建议对我的学术文章和毕业论文起到了很大的促进作用。

一路走来,
还有许多,
不再一一列举。
在这里感谢南航的王春武、朱君和尹晨老师等,
对诸位老师的帮忙和关怀表示诚挚的谢意!感谢帮助过我的师兄、师姐、师弟、师妹、朋友、室友和同学们,
感谢你们在生活上的照顾,
在学习上的帮助。
我也要感谢参考文献中的作者们,
透过他们的研究文章,
使我对研究课题有了充分的认识。
我还要谢谢论文评阅老师和答辩委员会成员们的辛苦工作。

最后,
衷心感谢我的家人们,
真是在他们的鼓励和支持下我才得以顺利完成博士期间的研究以及此论文。
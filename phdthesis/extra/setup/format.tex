% \allowdisplaybreaks[4]
\setlength{\parindent}{2em}
\arraycolsep=1.6pt

% 章 ########################################################
\setcounter{secnumdepth}{4}
\titleformat{\chapter}{\center\bfseries\xiaosan\song}{第\chinese{chapter}章\quad}{0em}{}
\makeatletter
\renewcommand{\chaptermark}[1]{%
  \if@mainmatter%
    \markboth{第\chinese{chapter}章\quad #1}{\chinesethesistitle}%
  \else%
    \markboth{#1}{\chinesethesistitle}%
  \fi%
}
\makeatother
\renewcommand{\sectionmark}[1]{}
\titlespacing{\chapter}{0pt}{-5.5mm}{8mm} % 小三,黑宋
\crefformat{chapter}{#2第#1章#3}
\crefrangeformat{chapter}{#3第#1#4~#5#2章#6}
\crefmultiformat{chapter}{#2第#1#3}{和#2#1章#3}{、#2#1#3}{和#2#1章#3}
\crefrangemultiformat{chapter}{#3第#1#4~#5#2#6}{和#3#1#4~#5#2章#6}{、#3#1#4~#5#2#6}{和#3#1#4~#5#2章#6}

% 节 ########################################################
\titleformat{\section}{\bfseries\sihao\song}{\thesection \quad}{0em}{}
\titlespacing{\section}{0pt}{4.5mm}{4.5mm} % 四号,黑宋
\crefformat{section}{#2第#1节#3}
\crefrangeformat{section}{#3第#1#4~#5#2节#6}
\crefmultiformat{section}{#2第#1#3}{和#2#1节#3}{、#2#1#3}{和#2#1节#3}
\crefrangemultiformat{section}{#3第#1#4~#5#2#6}{和#3#1#4~#5#2节#6}{、#3#1#4~#5#2#6}{和#3#1#4~#5#2节#6}

% 小节 ########################################################
\titleformat{\subsection}{\bfseries\xiaosi\song}{\thesubsection \quad}{0em}{}
\titlespacing{\subsection}{0pt}{4mm}{4mm} % 小四,黑宋
\crefformat{subsection}{#2第#1小节#3}
\crefrangeformat{subsection}{#3第#1#4~#5#2小节#6}
\crefmultiformat{subsection}{#2第#1#3}{和#2#1小节#3}{、#2#1#3}{和#2#1小节#3}
\crefrangemultiformat{subsection}{#3第#1#4~#5#2#6}{和#3#1#4~#5#2小节#6}{、#3#1#4~#5#2#6}{和#3#1#4~#5#2小节#6}

% 项 ########################################################
\titleformat{\subsubsection}{\xiaosi\kai}{\thesubsubsection \quad}{0em}{}
\titlespacing{\subsubsection}{0pt}{0pt}{0pt} % 小四,楷体
\crefformat{subsubsection}{#2第#1项#3}
\crefrangeformat{subsubsection}{#3第#1#4~#5#2项#6}
\crefmultiformat{subsubsection}{#2第#1#3}{和#2#1项#3}{、#2#1#3}{和#2#1项#3}
\crefrangemultiformat{subsubsection}{#3第#1#4~#5#2#6}{和#3#1#4~#5#2项#6}{、#3#1#4~#5#2#6}{和#3#1#4~#5#2项#6}

\makeatletter
\def\@dotsep{0.75} % 定义目录的点间距
\renewcommand*\l@chapter{\@dottedtocline{0}{0em}{4em}}% 控制目录:细点\@dottedtocline 粗点\@dottedtoclinebold
\renewcommand*\l@section{\@dottedtocline{1}{0em}{1.8em}} % 目录无缩进
\renewcommand*\l@subsection{\@dottedtocline{2}{0em}{2.5em}} % 目录无缩进

\urlstyle{same} % 论文中引用的网址的字体默认与正文中字体不一致,这里修正为一致的。

% 调整罗列环境的布局
\setitemize{labelsep=0.5em,wide,itemindent=\parindent-0.5em,parsep=0em,rightmargin=0em,itemsep=0em,topsep=0em,partopsep=0.3\baselineskip}
\setenumerate{labelsep=0.5em,wide,itemindent=\parindent+0.5em,parsep=0em,rightmargin=0em,itemsep=0em,topsep=0em,partopsep=0.3\baselineskip}

% 定制浮动图形和表格标题样式
\captionnamefont{\bfseries\wuhao}
\captiontitlefont{\bfseries\wuhao}
\captiondelim{~~}
\captionstyle{\centering}
\renewcommand{\subcapsize}{\wuhao}
\setlength{\abovecaptionskip}{1ex}
\setlength{\belowcaptionskip}{1ex}

% 默认字体
\renewcommand\normalsize{
  \@setfontsize\normalsize{12pt}{12pt}
  \setlength\abovedisplayskip{4pt}
  \setlength\abovedisplayshortskip{4pt}
  \setlength\belowdisplayskip{4pt}
  \setlength\belowdisplayshortskip{4pt}
  \let\@listi\@listI}
% 默认五号字体
\newcommand\normalsmallsize{
  \@setfontsize\normalsize{10.5pt}{10.5pt}
  \setlength\abovedisplayskip{4pt}
  \setlength\abovedisplayshortskip{4pt}
  \setlength\belowdisplayskip{4pt}
  \setlength\belowdisplayshortskip{4pt}
  \let\@listi\@listI}

% 设置行距和段落间垂直距离,根据规定,20磅
\def\defaultfont{\renewcommand{\baselinestretch}{1.64}\normalsize\selectfont}
\def\defaultfontsmall{\renewcommand{\baselinestretch}{1.64}\normalsmallsize\selectfont}

\renewcommand{\CJKglue}{\hskip 0.56pt plus 0.08\baselineskip} %加大字间距,使每行36个字。
\ziju{0.06}
% \predisplaypenalty=0 %公式之前可以换页,公式出现在页面顶部

\newcommand{\ClearDoubleEmptyPage}{\clearpage{\pagestyle{empty}\cleardoublepage}}
\preto{\chapter}{\ClearDoubleEmptyPage}

\renewcommand{\headrule}{} % 去除页眉的横线
\pagestyle{empty} % 设置封面部分为空的页眉页脚
\apptocmd{\@chapter}{\thispagestyle{empty}}{}{}
\apptocmd{\@schapter}{\thispagestyle{empty}}{}{}

\fancypagestyle{firstpage}{
  \fancyhead{} % 清空页眉
  \fancyfoot[OR,EL]{\wuhao \thepage} % 页脚:页码
}
\fancypagestyle{otherpage}{
  \fancyhead[CE]{
    \song \wuhao \leftmark\\
    \rule[0.7\baselineskip]{\headwidth}{.5pt}
  } % 正文页眉
  \fancyhead[CO]{
    \song \wuhao \rightmark\\
    \rule[0.7\baselineskip]{\headwidth}{.5pt}
  } % 正文页眉
  \fancyfoot[OR,EL]{\wuhao \thepage} % 页脚:页码
}
\renewcommand\frontmatter{
  \ClearDoubleEmptyPage
  \@mainmatterfalse
  \pagenumbering{Roman}
  \fancyhf{}
  \pagestyle{otherpage}
  \apptocmd{\@chapter}{\thispagestyle{firstpage}}{}{}
  \apptocmd{\@schapter}{\thispagestyle{firstpage}}{}{}
}
\newcommand\appmatter{
  \appendix
  \titleformat{\chapter}{\center\bfseries\xiaosan\song}{附录\thechapter \quad}{0em}{}
  \renewcommand{\chaptermark}[1]{%
    \if@mainmatter%
      \markboth{附录\thechapter \quad ##1}{\chinesethesistitle}%
    \else%
      \markboth{##1}{\chinesethesistitle}%
    \fi%
  }
}

% 封面、摘要、版权、致谢格式定义
\def\ctitle#1{\def\@ctitle{#1}}\def\@ctitle{}
\def\cdegree#1{\def\@cdegree{#1}}\def\@cdegree{}
\def\csubject#1{\def\@csubject{#1}}\def\@csubject{}
\def\cauthor#1{\def\@cauthor{#1}}\def\@cauthor{}
\def\csupervisor#1{\def\@csupervisor{#1}}\def\@csupervisor{}
% \def\cassosupervisor#1{\def\@cassosupervisor{副导师:#1\\}}\def\@cassosupervisor{}
% \def\ccosupervisor#1{\def\@ccosupervisor{联合导师:#1}}\def\@ccosupervisor{}
\def\cassosupervisor#1{\def\@cassosupervisor{#1\\}}\def\@cassosupervisor{}
\def\ccosupervisor#1{\def\@ccosupervisor{}}\def\@ccosupervisor{}
\def\ccommitteechairman#1{\def\@ccommitteechairman{#1}}\def\@ccommitteechairman{}
\def\creviewers#1{\def\@creviewers{#1}}\def\@creviewers{}
\def\csdate#1{\def\@csdate{#1}}\def\@csdate{}
\def\cddate#1{\def\@cddate{#1}}\def\@cddate{}
\def\cdate#1{\def\@cdate{#1}}\def\@cdate{}
\long\def\cabstract#1{\long\def\@cabstract{#1}}\long\def\@cabstract{}
\def\ckeywords#1{\def\@ckeywords{#1}}\def\@ckeywords{}

\def\etitle#1{\def\@etitle{#1}}\def\@etitle{}
\def\edegree#1{\def\@edegree{#1}}\def\@edegree{}
\def\esubject#1{\def\@esubject{#1}}\def\@esubject{}
\def\eauthor#1{\def\@eauthor{#1}}\def\@eauthor{}
\def\esupervisor#1{\def\@esupervisor{#1}}\def\@esupervisor{}
% \def\eassosupervisor#1{\def\@eassosupervisor{Associate Supervisor: #1\\}}\def\@eassosupervisor{}
% \def\ecosupervisor#1{\def\@ecosupervisor{Co Supervisor: #1\\}}\def\@ecosupervisor{}
\def\eassosupervisor#1{\def\@eassosupervisor{#1\\}}\def\@eassosupervisor{}
\def\ecosupervisor#1{\def\@ecosupervisor{}}\def\@ecosupervisor{}
\def\ecommitteechairman#1{\def\@ecommitteechairman{#1}}\def\@ecommitteechairman{}
\def\ereviewers#1{\def\@ereviewers{#1}}\def\@ereviewers{}
\def\esdate#1{\def\@esdate{#1}}\def\@esdate{}
\def\eddate#1{\def\@eddate{#1}}\def\@eddate{}
\def\edate#1{\def\@edate{#1}}\def\@edate{}
\long\def\eabstract#1{\long\def\@eabstract{#1}}\long\def\@eabstract{}
\long\def\NotationList#1{\long\def\@NotationList{#1}}\long\def\@NotationList{}
\def\ekeywords#1{\def\@ekeywords{#1}}\def\@ekeywords{}
\def\classifiedindex#1{\def\@classifiedindex{#1}}\def\@classifiedindex{}
\def\unidecimalclass#1{\def\@unidecimalclass{#1}}\def\@unidecimalclass{}
\def\statesecrets#1{\def\@statesecrets{#1}}\def\@statesecrets{}
\def\thesisnum#1{\def\@thesisnum{#1}}\def\@thesisnum{}

% 定义封面
\def\makecover{
  \begin{titlepage}
    % 封面一
    \begin{center}

      {\xiaosi
        \begin{tabular}{@{}r@{:}l@{}}
          分类号    & \uline{\makebox[51mm]{\@classifiedindex}} \\
          U.D.C. & \uline{\makebox[51mm]{\@unidecimalclass}}
        \end{tabular}}\hfill
      {\xiaosi
        \begin{tabular}{@{}r@{:}l@{}}
          密级 & \uline{\makebox[51mm]{\@statesecrets}} \\
          编号 & \uline{\makebox[51mm]{\@thesisnum}}
        \end{tabular}}
      \parbox[t][36pt][t]{\textwidth}{\begin{center} \end{center} }
      
      \parbox[t][44pt][t]{\textwidth}{\chuhao
        \begin{center} {\song \bfseries 中国工程物理研究院} \end{center} }
      \parbox[t][44pt][t]{\textwidth}{\begin{center} \end{center} }
      \begin{center} {\song \yihao \bfseries 学~~位~~论~~文 }\end{center}
      
      \kai
      
      \parbox[t][70pt][t]{\textwidth}{ \bfseries
        \begin{center} \renewcommand{\arraystretch}{2.0}
          \begin{tabular}{c}
            \xiaoer \uline{\makebox[120mm]{\kai \StrMid{\@ctitle}{1}{\ctitlefirstend}}}       \\
            \xiaoer \uline{\makebox[120mm]{\kai \StrMid{\@ctitle}{\ctitlesecondstart}{1000}}} \\
            \sanhao \uline{\makebox[20mm]{\song\@cauthor}}
          \end{tabular} \renewcommand{\arraystretch}{1}
        \end{center} }
      \parbox[t][40pt][t]{\textwidth}{\begin{center} \end{center} }
      
      \parbox[t][150pt][t]{\textwidth}{\bfseries
        \begin{center} \renewcommand{\arraystretch}{2.0} \song \xiaosi
          \begin{tabular}{r}
            {\kai \sihao 指导教师姓名} \uline{\makebox[105.7mm]{\@csupervisor}}                                                 \\
            \uline{\makebox[105.7mm]{\@cassosupervisor \@ccosupervisor}}                                                  \\
            {\kai \sihao 申请学位级别} \uline{\makebox[36mm]{\@cdegree}} {\kai \sihao 专业名称} \uline{\makebox[47mm]{\@csubject}}  \\
            {\kai \sihao 论文提交日期} \uline{\makebox[36mm]{\@csdate}} {\kai \sihao 论文答辩日期} \uline{\makebox[36.6mm]{\@cddate}} \\
            {\kai \sihao 授予学位单位} \uline{\makebox[105.7mm]{中国工程物理研究院}}                                                     \\
            {\kai \sihao 答辩委员会主席} \uline{\makebox[45mm]{\@ccommitteechairman}}                                            \\
            %{\kai \sihao 评阅人\uline{\parbox[b]{45mm}{\@creviewers}}}
          \end{tabular} \renewcommand{\arraystretch}{1}
        \end{center} }
      \parbox[t][130pt][t]{\textwidth}{\begin{center} \end{center} }
      
      {\xiaosi \bfseries \@cdate}
    \end{center}
    
    % 英文封面
    \ClearDoubleEmptyPage
    \begin{center}
      
      \vspace*{3cm}
      {
        \Arial
        \fontsize{20pt}{20pt}\selectfont
        \bfseries
        \@etitle
      }

      \vspace{\stretch{3}}

      \sanhao
      Dissertation Submitted to

      {\bfseries China Academy of Engineering Physics}

      in partial fulfillment of the requirement 

      for the degree of 

      {\Arial\bfseries\@edegree}

      \vspace{\stretch{1}}

      in

      {\Arial\bfseries\@esubject}

      \vspace{\stretch{1}}

      by

      {\Arial\bfseries\@eauthor}

      \vspace{\stretch{2}}

      \xiaosan
      Dissertation Supervisor:
      \@esupervisor{}

      \vspace{\stretch{3}}

      {\sanhao\Arial\bfseries\@edate}
    \end{center}
  \end{titlepage}
}

\makeatother